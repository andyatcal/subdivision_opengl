\documentclass[12pt]{article}

\title{An Implementation of Halfedge Data Structure in Catmull-Clark Subdivision for 2-Manifold Single-sided Surface}
\author{Yu Wang}
\date{August 2015}

\begin{document}
\maketitle
\newpage

%\begin{abstract} A place for abstract later
% Contents of abstract
%\end{abstract}

\section{Introduction}

\section{Halfedge Data Structure} \label{sec:halfedge}

An object in 3D Euclid space can be represented by a mesh of polygons. A mesh contains three types of geometry element: vertex, edge, and face. The adjacency structure stores the topological information (adjacency and connectivity) of the mesh. The author chose halfedge data structure as the adjacency structure in this project to realize Catmull-Clark subdivision.

\subsection{Vertex, Halfedge, and Face}

The definitions and assumptions of vertex, halfedge and face are shown in table \ref{table:vhfdef}. An example for a single face is shown in figure.
\begin{table}[h]
\centering
\begin{tabular}{| l | p{0.4\textwidth} | p{0.4\textwidth}|}
\hline
		&	Definition	& Assumption	\\
\hline
Vertex	&	A 3-dimensional point		&	Vertices can overlap when they have the same position		\\
\hline
Halfedge	&	An edge that starts from one vertex and end at another vertex & A halfedge connects exactly two non-overlapping vertices and it has a direction\\
\hline
Face		&	A polygon that contains a loop of vertices and halfedges	& A face has at least three non-overlapping vertices so it makes a polygon\\
\hline
\end{tabular}
\caption{Definitions and assumptions of vertex, halfedge, and face} 
\label{table:vhfdef}
\end{table}

\subsection{Face Connections}

\section{Catumll-Clark Subdivision} \label{sec:ccsd}

\subsection{General Approach of Catmull-Clark Subdivision}

\subsubsection{Compute Vertex Positions of New Mesh}

\subsubsection{Make Connections of New Mesh}

\subsection{Sharp Crease and Boundary Feature}

\subsection{Mobius Connection}

\section{Offset Surface} \label{sec:offset}

\subsection{Compute Vertex Normals}

\subsection{Positive and Negative Offsets}

\subsection{Mobius Connection Issue}


\section{Test Cases and Discussions}



\section{Contribution to Knowledge}





\section{Future Researches}


\end{document}